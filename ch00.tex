%!TEX root = da-dev.tex

In the analysis of distributed algorithms, we will sometimes encounter power towers and iterated logarithms.


\usection{Power Tower}

We write power towers with the notation
\[
    {}^i 2 = 2^{2^{\cdot^{\cdot^2}}},
\]
where there are $i$ twos in the tower. Power towers grow very fast; for example,
\begin{align*}
    {}^1 2 &= 2,\\
    {}^2 2 &= 4,\\
    {}^3 2 &= 16,\\
    {}^4 2 &= 65536,\\
    {}^5 2 &= 2^{65536} > 10^{19728}.
\end{align*}


\usection{Iterated Logarithm}

The iterated logarithm of $x$, in notation $\log^* x$, is defined recursively as follows:
\[
    \log^*(x) = \begin{cases}
        0 & \text{ if $x \le 1$}, \\
        1 + \log^*(\log_2 x) & \text{ otherwise}.
    \end{cases}
\]
In essence, this is the inverse of the power tower function. For all positive integers $i$, we have
\[
    \log^* {}^i 2 = i.
\]
As power towers grow very fast, iterated logarithms grow very slowly; for example,
\begin{align*}
    \log^* 2 &= 1, &
    \log^* 16 &= 3, &
    \log^* 10^{10} &= 5, \\
    \log^* 3 &= 2, &
    \log^* 17 &= 4, &
    \log^* 10^{100} &= 5, \\
    \log^* 4 &= 2, &
    \log^* 65536 &= 4, &
    \log^* 10^{1000} &= 5, \\
    \log^* 5 &= 3, &
    \log^* 65537 &= 5, &
    \log^* 10^{10000} &= 5, \dotsc
\end{align*}

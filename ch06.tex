%!TEX root = da-dev.tex

FIXME: $\CONGEST$ model

\section{Exercises}

\begin{ex}[prior algorithms]
    In Chapters \ref{ch:pn} and \ref{ch:local} we have seen examples of algorithms that were designed for the $\PN$ and $\LOCAL$ models. Many of these algorithms use only small messages\mydash they can be used directly in the $\CONGEST$ model. Give at least three examples of such algorithms.
\end{ex}

\begin{ex}[edge counting]
    The \emph{edge counting} problem is defined as follows: each node has to output the value $|E|$, i.e., it has to indicate how many edges there are in the graph.

    Assume that the input graph is connected. Design an algorithm that solves the edge counting problem in the $\CONGEST$ model in time $O(\diam(G))$.
\end{ex}

\begin{ex}[detecting bipartite graphs]
    Assume that the input graph is connected. Design an algorithm that solves the following problem in the $\CONGEST$ model in time $O(\diam(G))$:
    \begin{itemize}[noitemsep]
        \item If the input graph is bipartite, all nodes output $1$.
        \item Otherwise all nodes outputs $0$.
    \end{itemize}
\end{ex}

\begin{ex}[detecting complete graphs]
    We say that a graph $G = (V,E)$ is \emph{complete} if for all nodes $u, v \in V$, $u \ne v$, there is an edge $\{u,v\} \in E$.

    Assume that the input graph is connected. Design an algorithm that solves the following problem in the $\CONGEST$ model in time $O(1)$:
    \begin{itemize}[noitemsep]
        \item If the input graph is a complete graph, all nodes output $1$.
        \item Otherwise all nodes output $0$.
    \end{itemize}
\end{ex}

\begin{ex}[gathering]
    Assume that the input graph is connected. In Section~\ref{sec:gather} we saw how to gather full information on the input graph in time $O(\diam(G))$ in the $\LOCAL$ model. Design an algorithm that solves the problem in time $O(|E|)$ in the $\CONGEST$ model.
\end{ex}

\begin{exs}[lower bounds]
    Prove that there is no algorithm that gathers full information on the input graph in time $O(\diam(G))$ in the $\CONGEST$ model.

    \hint{To reach a contradiction, assume that $A$ is an algorithm that solves the problem. Fix a diameter $d$, e.g., $d = 4$. Let $T$ be the running time of algorithm~$A$ in graphs of diameter~$d$. Consider the family $\calF$ of graphs with the following properties: there is a node $v$ with unique identifier $1$, the degree of node $v$ is $1$, and the diameter of the graph is $d$. Then compare the following two quantities:
    \begin{enumerate}
        \item How many different graphs there are in family $\calF$.
        \item How many different message sequences node $v$ may receive during the execution of algorithm~$A$.
    \end{enumerate}
    Then argue that there are at least two different graphs $G_1, G_2 \in \calF$ such that node $v$ receives the same information when we run $A$ in either of these graphs.}
\end{exs}

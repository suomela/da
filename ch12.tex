%!TEX root = da-screen.tex

FIXME

\section{Further Reading}

Nancy Lynch's textbook~\cite{lynch96book} provides an excellent overview of the field of distributed algorithms. Diestel's book~\cite{diestel05graph} is a good source for graph-theoretic background, and Vazirani's book~\cite{vazirani01approximation} provides further information on approximation algorithms from the perspective of non-distributed computing.

For more online material on distributed algorithms, see the following web page:
\begin{quote}
    Principles of Distributed Computing, \\
    Distributed Computing Group, ETH Zurich \\[1ex]
    \url{http://dcg.ethz.ch/lectures/podc_allstars/}
\end{quote}


\section{Bibliographic Notes}

Many parts of this course have been directly influenced by numerous papers and textbooks; here is a brief summary of the key references.

\paragraph{Graph-theoretic Foundations.}

The connection between minimum maximal matchings and minimum edge dominating sets (Exercise~\ref{ex:mmeds}) is due to Allan and Laskar~\cite{allan78domination} and Yannakakis and Gavril~\cite{yannakakis80edge}, and the connection between maximal edge packings and approximations of vertex covers (Lemma~\ref{lem:mep-vc}) was identified by Bar-Yehuda and Even \cite{bar-yehuda81linear-time}. The connection between maximal matchings and approximations of vertex covers (Exercise~\ref{ex:mmvc}) is commonly attributed to Gavril and Yannakakis (see, e.g., Papadimitriou and Steiglitz~\cite{papadimitriou98combinatorial}). Exercise~\ref{ex:2fact} is a 120-year-old result due to Petersen~\cite{petersen1891dietheorie}. The definition of a weak colouring is from Naor and Stockmeyer~\cite{naor95what}. Ramsey's theorem dates back to 1930s~\cite{ramsey30problem}; our proof follows Ne{\v s}et{\v r}il~\cite{nesetril95ramsey}, and the notation is from Radziszowski~\cite{radziszowski11ramsey}.

\paragraph{Model of Computing.}

The model of computing that we use throughout this course\mydash running time equals the number of synchronous communication rounds\mydash is from Linial's~\cite{linial92locality} seminal paper, while the concept of a port numbering is from Angluin's~\cite{angluin80local} work.

\paragraph{Algorithms.}

Algorithm $\algo{DPBit}$ is based on the idea originally introduced by Cole and Vishkin~\cite{cole86deterministic} and further refined by Goldberg et al.~\cite{goldberg88parallel}. The idea of algorithm $\algo{DPSet}$ is from Naor and Stockmeyer~\cite{naor95what}. Algorithm $\algo{Colour}$ is from Goldberg et al.~\cite{goldberg88parallel} and Panconesi and Rizzi~\cite{panconesi01some}. Algorithm $\algo{BMM}$ is due to Ha\'{n}\'{c}kowiak et al.~\cite{hanckowiak98distributed}. Algorithm of Exercise~\ref{ex:greedy-domset} is from Friedman and Kogan~\cite{friedman11deterministic}.

\paragraph{Lower Bounds.}

The use of covering maps in the context of distributed algorithm was introduced by Angluin~\cite{angluin80local}, and local neighbourhoods were studied by, among others, by Linial~\cite{linial92locality}. The general idea of Exercise~\ref{ex:cover-three-reg-b} can be traced back to Yamashita and Kameda~\cite{yamashita96computing}, while the specific construction in Figure~\ref{fig:cover-ex-three-reg-b} is from Bondy and Murty's textbook~\cite[Figure~5.10]{bondy76graph-theory}. Lower bounds on graph colouring in the model of unique identifiers are from Linial's seminal work~\cite{linial92locality}; our presentation in Section~\ref{sec:speed-limits} uses an alternative proof based on Ramsey's theorem, following, e.g., Naor and Stockmeyer~\cite{naor95what} and Czygrinow et al.~\cite{czygrinow08fast}. In particular, the idea of Exercise~\ref{ex:is-apx-lb} is from Czygrinow et al.~\cite{czygrinow08fast}.

\paragraph{Local Work.}

Recent work by our research group is represented in algorithms $\algo{VC3}$~\cite{polishchuk09simple} and $\algo{VC2}$~\cite{astrand09vc2apx}. Many exercises are also inspired by our work, including Exercises \ref{ex:cover-three-reg1} and~\ref{ex:cover-four-reg} \cite{suomela10eds}, Exercise~\ref{ex:cover-complete} \cite{astrand10weakly-coloured}, Exercise~\ref{ex:domset} \cite{astrand10weakly-coloured}, and Exercises \ref{ex:edsfirst}--\ref{ex:edslast}~\cite{suomela10eds}.


\section{Exercises}

In the following exercises, we will study distributed approximation algorithms for the edge dominating set problem. We will first show that the problem is easy to approximate within factor $4$ in general graphs. Then we will have a look at some special cases, and derive tight upper and lower bounds for the approximation ratio. We use the abbreviation \emph{MEDS} for a minimum edge dominating set.

\begin{ex}[general case]\label{ex:edsfirst}
    Design a $\PN$-algorithm that finds a \Apx{4} of MEDS.
    
    \hint{Use the idea of Section~\ref{ssec:vc3}. Show that the edge set $M \subseteq E$ defined in \eqref{eq:vc3-M} is a $4$-approximation of MEDS. To this end, consider an optimal solution $D^*$ and show that each edge of $D^*$ is adjacent to at most $4$ edges of~$M$.}
\end{ex}

\begin{ex}[\Reg{2}, upper bounds]
    Show that the following is possible in \Reg{2} graphs:
    \begin{subex}
        \item finding a \Apx{3} of MEDS in $O(1)$ time in the $\PN$ model
        \item finding a \Apx{2} of MEDS in $O(\log^* n)$ time in the $\LOCAL$ model
        \item finding a \Apx{2} of MEDS with a randomised algorithm in the $\PN$ model
    \end{subex}
\end{ex}

\begin{ex}[\Reg{2}, lower bounds]
    Show that the following is not possible in \Reg{2} graphs:
    \begin{subex}
        \item finding a \Apx{2.999} of MEDS in the $\PN$ model
        \item finding a \Apx{2.999} of MEDS in $O(1)$ time in the $\LOCAL$ model
    \end{subex}
\end{ex}

\begin{ex}[\Reg{4}, upper bound]
    Show that it is possible to find a \Apx{3.5} of MEDS in \Reg{4} graphs in constant time in the $\PN$ model.
    
    \hint{Consider an algorithm that selects all edges that have port number $1$ in at least one end. Derive an upper bound on the size of the solution and a lower bound on the size of an optimal solution, as a function of $|V|$.}
\end{ex}

\begin{ex}[\Reg{4}, lower bound]
    Show that it is not possible to find a \Apx{3.499} of MEDS in \Reg{4} graphs in the $\PN$ model.
    
    \hint{Use the construction of Exercise~\ref{ex:cover-four-reg}.}
\end{ex}

\begin{ex}[\Reg{3}, lower bound]
    Show that it is not possible to find a \Apx{2.499} of MEDS in \Reg{3} graphs in the $\PN$ model.
    
    \hint{Use the construction of Exercise~\ref{ex:cover-three-reg1}.}
\end{ex}

\begin{ex}[\Reg{3}, upper bound]\label{ex:edslast}
    Show that it is possible to find a \Apx{2.5} of MEDS in \Reg{3} graphs in constant time in the $\PN$ model.
    
    \hint{Let $G = (V,E)$ be a $3$-regular graph. We say that a set $D \subseteq E$ is \emph{good} if it satisfies the following properties:
    \begin{enumerate}
        \item $D$ is an edge cover for $G$,
        \item the subgraph induced by $D$ does not contain a path of length $3$.
    \end{enumerate}
    Put otherwise, $D$ induces a spanning subgraph that consists of node-disjoint stars. Prove that
    \begin{enumerate}
        \item any good set $D$ is a \Apx{2.5} of MEDS,
        \item there is a distributed algorithm that finds a good set $D$.
    \end{enumerate}
    The distributed algorithm has to exploit the port numbers of the edges. One possible approach is this: First, use the port numbers to find nine matchings, $M_1, M_2, \dotsc, M_9$, such that each node is incident to an edge in at least one of the sets $M_i$; do not worry if some edges are present in more than one matching. Then construct an edge cover $D$ by greedily adding edges from the sets $M_i$; in step $i = 1, 2, \dotsc, 9$ you can consider all edges of $M_i$ in parallel. Finally, eliminate paths of length three by removing redundant edges in order to make $D$ a good set; again, in step $i = 1, 2, \dotsc, 9$ you can consider all edges of $M_i$ in parallel.}
\end{ex}

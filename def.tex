%!TEX root = da-dev.tex

%---- Sections ----

\newcommand{\uchapter}[1]{\chapter*{#1}\phantomsection\addcontentsline{toc}{chapter}{#1}}
\newcommand{\usection}[1]{\section*{#1}\phantomsection\addcontentsline{toc}{section}{#1}}

%---- Lists ----

\setenumerate{label=(\alph*)}

\newlist{descriptionb}{description}{1}
\setlist[descriptionb]{font=\normalfont\itshape,leftmargin=0pt,itemsep=1ex,style=unboxed}

\newlist{notation}{description}{1}
\setlist[notation]{font=\normalfont,labelindent=0em,leftmargin=6em,itemsep=0ex,style=sameline}

\newlist{algorithms}{description}{1}
\setlist[algorithms]{font=\normalfont,labelindent=0em,leftmargin=6em,itemsep=0ex,style=sameline}

\newlist{subex}{enumerate}{1}
\setlist[subex]{label=(\alph*),itemsep=0.25ex}

\newcommand{\algtoprule}{\rule{\columnwidth}{\heavyrulewidth}{}}
\newcommand{\algbottomrule}{\rule[1ex]{\columnwidth}{\heavyrulewidth}{}}

%---- Theorems ----

\newtheorem{theorem}{Theorem}[chapter]
\newtheorem{lemma}[theorem]{Lemma}
\newtheorem{corollary}[theorem]{Corollary}

\theoremstyle{definition}
\newtheorem{ex}{Exercise}[chapter]
\newtheorem{exs}[ex]{$\star$ Exercise}

%---- Notation ----

\DeclareMathOperator{\ball}{ball}
\DeclareMathOperator{\diam}{diam}
\DeclareMathOperator{\dist}{dist}
\DeclareMathOperator{\outdegree}{outdegree}
\DeclareMathOperator{\indegree}{indegree}

\DeclareMathOperator{\Input}{Input}
\DeclareMathOperator{\Output}{Output}
\DeclareMathOperator{\States}{States}
\DeclareMathOperator{\Msg}{Msg}
\DeclareMathOperator{\Init}{init}
\DeclareMathOperator{\Send}{send}
\DeclareMathOperator{\Receive}{receive}
\DeclareMathOperator{\Id}{id}

\renewcommand{\emptyset}{\varnothing}
\newcommand{\calF}{\mathcal{F}}
\newcommand{\calS}{\mathcal{S}}
\newcommand{\NN}{\mathbb{N}}
\newcommand{\RR}{\mathbb{R}}

\newcommand{\PN}{\textsf{PN}}
\newcommand{\LOCAL}{\textsf{LOCAL}}
\newcommand{\CONGEST}{\textsf{CONGEST}}
\newcommand{\tmodel}[1]{\texorpdfstring{{\sffamily #1}}{#1}}
\newcommand{\tPN}{\tmodel{PN}}
\newcommand{\tLOCAL}{\tmodel{LOCAL}}
\newcommand{\tCONGEST}{\tmodel{CONGEST}}

\newcommand{\algo}[1]{\textsf{#1}}
\newcommand{\state}[1]{\textsf{\small #1}}
\newcommand{\msg}[1]{`\emph{#1}'}
\newcommand{\bin}[1]{\langle #1 \rangle}

%---- Macros ----

\newcommand{\Set}[1]{\{\, #1 \,\}}
\newcommand{\bigSet}[1]{\bigl\{\, #1 \,\bigr\}}
\newcommand{\Apx}[1]{$#1$\hyp approximation}
\newcommand{\Fact}[1]{$#1$\hyp factorisation}
\newcommand{\Reg}[1]{$#1$-regular}
\newcommand{\Dpocol}{$(\Delta+\nobreak 1)$-colouring}
\newcommand{\mydash}{ --- }
\newcommand{\mysep}{\begin{center}$* \quad * \quad *$\end{center}}

%---- Figures ----

\newcounter{myexternalpagenum}
\newcommand{\definepage}[1]{\stepcounter{myexternalpagenum}\edef#1{\arabic{myexternalpagenum}}}
\definepage{\PGraph}
\definepage{\PWalk}
\definepage{\PCycle}
\definepage{\PNeighbourhood}
\definepage{\PIndependentSet}
\definepage{\PPackingCovering}
\definepage{\PPartitions}
\definepage{\PFactorisation}
\definepage{\PPnnA}
\definepage{\PPnnB}
\definepage{\PPnnC}
\definepage{\PPnnD}
\definepage{\PMaximalMatching}
\definepage{\PVCThreeApx}
\definepage{\PVCThreeApxB}
\definepage{\PVCThreeApxC}
\definepage{\PPnnTwoNode}
\definepage{\PCoveringMap}
\definepage{\PCoveringMapB}
\definepage{\PCoveringMapC}
\definepage{\PThreeReg}
\definepage{\PCyclesAndCovers}
\definepage{\PTwoEdgeCol}
\definepage{\PPathSameNeigh}
\definepage{\PSameNeigh}
\definepage{\PCoverExFourReg}
\definepage{\PCoverExThreeReg}
\definepage{\PCoverExThreeRegSpoiler}
\definepage{\PCoverExThreeRegB}
\definepage{\PNeighEx}
\definepage{\PMep}
\definepage{\PMepB}
\definepage{\PHalfSaturating}
\definepage{\PHalfSaturatingB}
\definepage{\PHalfSaturatingC}
\definepage{\PUniqueIds}
\definepage{\PGather}
\definepage{\PGreedy}
\definepage{\PDP}
\definepage{\PDPSuccessor}
\definepage{\PDPGreedy}
\definepage{\PDPCritical}
\definepage{\PIdOrient}
\definepage{\PIdPickClass}
\definepage{\PMergeColours}
\definepage{\PDirectedCycle}
\definepage{\PRtoGPart}
\definepage{\PSubsetCycle}
\definepage{\PColourLB}
\definepage{\PIntroTopo}
\definepage{\PIntroCol}
\definepage{\PIntroColTwo}
\definepage{\PIntroDegOne}
\definepage{\PIntroDegTwo}
\definepage{\PIntroTwo}
\definepage{\PIntroId}
\definepage{\PIntroIdA}
\definepage{\PIntroIdAA}
\definepage{\PIntroIdB}
\definepage{\PIntroIdBB}
\definepage{\PIntroIdC}
\definepage{\PIntroIdCC}
\definepage{\PIntroIdD}
\definepage{\PIntroIdDD}
\definepage{\PIntroIdE}
\definepage{\PIntroIdEE}
\definepage{\PIntroIdBad}
\definepage{\PIntroIdDir}
\definepage{\PIntroMis}
\definepage{\PIntroIdX}
\definepage{\PIntroTA}
\definepage{\PIntroTB}
\definepage{\PIntroTC}
\definepage{\PIntroTD}
\definepage{\PIntroTDB}
\definepage{\PIntroTwoColA}
\definepage{\PIntroTwoColB}
\definepage{\PIntroLbTwoA}
\definepage{\PIntroLbTwoB}
\definepage{\PIntroLbTwoC}
\definepage{\PIntroLbTwoD}
\definepage{\PIntroLbTwoE}


\hyphenation{auton-o-mous Hir-vo-nen Ju-ho Hir-vo-nen Kaa-si-nen Suo-me-la Tuo-mo Wawr-zyniak Ya-ma-shi-ta}

%---- Hints ----

\setenotez{
    list-name = {Hints},
    backref = true,
    mark-cs = {\hintnumber},
    counter-format = Alph,
    totoc = chapter
}

\DeclareInstance{enotez-list}{custom}{list}
{
    format = \normalfont,
    number-format = \normalfont,
    list-type = itemize
}

\newcommand{\hint}[1]{\nopagebreak\par\hfill{\footnotesize\em $\triangleright$ hint \endnote{#1}\par}}
\newcommand{\hintnumber}[1]{#1}
